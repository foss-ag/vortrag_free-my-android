\begin{frame}{Das Flashen - ADB}
	\begin{itemize}[<+->]
		\item Die \textbf{A}ndroid \textbf{D}ebug \textbf{B}ridge ist ein wichtiges Werkzeug
		\item Sie ermöglicht das Arbeiten und Kommunizieren mit dem Android System über eine USB oder WLAN Verbindung
		\item Mit ihr wird es möglich Dateien von und zum Rechner zu verschieben.
		\item Die \textbf{ADB} wird insbesondere zum Übertragen neuer CustomROMs verwendet, es gibt aber auch weitere Möglichkeiten.
		\begin{itemize}[<+->]
			\item Übertragen und Installieren von Apps mit dem Rechner
			\item Verwenden einer Shell vom Rechner aus
		\end{itemize}
	\end{itemize}
\end{frame}

\begin{frame}{Das Flashen - Fastboot}
	\begin{itemize}[<+->]
		\item Das \textbf{Fastboot}-Tool arbeitet eine Ebene tiefer im System
		\item Statt mit dem laufenden Android OS zu kommunizieren arbeitet es direkt auf der Firmware des Geräts
		\item Fastboot ermöglicht bei vielen Geräten das Entsperren des Bootloader-Sektors
		\begin{itemize}[<+->]
			\item Diese Funktion wird zum Flashen von Android-Derivaten benötigt
		\end{itemize}
		\item Außerdem können mit Fastboot direkt Images geflashed werden
	\end{itemize}
\end{frame}

\begin{frame}{Das Flashen - Mehr braucht es nicht?}
	\begin{itemize}[<+->]
		\item Die zuvor vorgestellten Tools reichen theoretisch zum Aufspielen alternativer Software aus
		\item Angepasste Android Versionen haben aber unter Umständen andere Schutzmechanismen
		\note{ Leider haben in der Praxis viele Hersteller eigene Wege entwickelt ihre Bootloader vor Zugriff zu schützen oder ein Aufspielen anderer Software zu limitieren }
		\item Daher braucht man je nach Gerät andere Software
		\item Wir gehen zusätzlich nur auf Samsung ein
		\note{ Der Verbreitung der Marke geschuldet gehen }
	\end{itemize}
\end{frame}

\begin{frame}{Das Flashen - Heimdall/Odin}
	\begin{itemize}[<+->]
		\item Heimdall und Odin sind für Samsung-Geräte erstellte Software
		\item Sie ersetzen Fastboot in ihrer Funktion
		\item Während Odin ein sog. Leak ist, wird Heimdall frei entwickelt.
	\end{itemize}
\end{frame}
