\begin{frame}{Android-Derivate}
\begin{columns}
	\begin{column}{0.5\textwidth}
		\begin{figure}
			\includegraphics[width=0.9\textwidth]{resources/Lineage_OS_Logo.png}
			%https://de.wikipedia.org/wiki/Linux_Mint#/media/File:Linux_Mint_logo_and_wordmark.svg
		\end{figure}
		
		\begin{figure}
			\includegraphics[width=0.8\textwidth]{resources/aokp-logo-large.png}
			%https://de.wikipedia.org/wiki/OpenSUSE#/media/File:OpenSUSE_Logo.svg
		\end{figure}
		
	\end{column}
	\begin{column}{0.5\textwidth}
		
		\begin{figure}
			\includegraphics[scale=2]{resources/PA-Head.png}
			%https://de.wikipedia.org/wiki/Ubuntu#/media/File:Ubuntu_logo.svg
		\end{figure}
		
		\begin{figure}
			\includegraphics[scale=0.15]{resources/Replicant_logo_alpha.png}
			%https://de.wikipedia.org/wiki/Fedora_%28Linux-Distribution%29#/media/File:Fedora_logo_and_wordmark.svg
		\end{figure}
		
	\end{column}
\end{columns}
\end{frame}

\begin{frame}{LineageOS}
	\begin{figure}
		\includegraphics[scale=0.25]{resources/Lineage_OS_Logo.png}
	\end{figure}
	\begin{itemize}[<+->]
		\item Erwuchs Weihnachten 2016 aus dem eingestellten CyanogenMod (\textbf{CM})
		\item CM war bis zu diesem Zeitpunkt das am weitesten verbreitete und unterstützte CustomROM
	\end{itemize}
\end{frame}

\begin{frame}{LineageOS}
	\begin{itemize}[<+->]
		\item LineageOS wird nun ausschließlich von der Community und einigen Entwicklern auf GitHub entwickelt
		\item Es unterstützt bis heute die meisten Geräte und bringt viele zusätzliche Features
		\item Daher werden wir am Ende des Vortrages exemplarisch ein LineageOS How-To-Flash für ein OnePlus One durchführen
	\end{itemize}
\end{frame}