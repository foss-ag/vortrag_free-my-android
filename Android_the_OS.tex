\begin{frame}{Android ist Hardware-Spezifisch}
	\begin{itemize}[<+->]
		\item Jedes System muss auf die gegebene Hardware des Gerätes angepasst werden
		\begin{itemize}
			\item Verhindert einen übergroßen Kernel
			\item Spart Speicher
			\item Erhöht Performance
		\end{itemize}
		\item Das verkompliziert das Anpassen von CustomROMs da die jeweiligen Geräte zum Testen benötigt werden
	\end{itemize}
\end{frame}

\begin{frame}{Struktur der Software}
	\begin{itemize}[<+->]
		\item Android teilt sich in folgende Partitionen:
		\begin{itemize}
			\item Bootloader 
			\item Recovery
			\item System
			\item Data
			\item SDCard / ExtSDCard
			\item Cache
		\end{itemize}
		\item Dazu kommen kleinere Partitionen welche weitere Systemfunktionen garantieren.
	\end{itemize}
	\pnote{ Bootloader: Lädt den Kernel und das schlussendliche System}
	\pnote{ Recovery: Ist ein "Rettungssystem", welches erweiterte Funktionen auf dem Gerät zur Verfügung stellen kann. (Backups, zusätzliche Pakete installieren, etc.)}
	\pnote{ System: Das Betriebssystem an sich}
	\pnote{ Data: Benutzerdatenpartition; Hier werden Kontakte, Nachrichten, Dauerhafte Appdaten und mehr gelagert}
	\pnote{ SDCard: Zusätzlicher Speicher der dem Nutzer frei zur Verfügung steht. Hier werden heruntergeladene Dateien, Musik, Bilder und Videos und weiteres gespeichert.}
	\pnote{ Cache: Partition zum Speichern temporärer Daten. Kann bedenkenlos geleert werden.}
\end{frame}