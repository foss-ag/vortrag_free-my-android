\begin{frame}{Android ist Hardware-Spezifisch}
	\begin{itemize}
		\item Jedes System muss auf die gegebene Hardware des Gerätes angepasst werden
		\begin{itemize}
			\item Verhindert einen übergroßen Kernel
			\item Spart Speicher
			\item Erhöht Performance
		\end{itemize}
		\item Das verkompliziert das Anpassen von CustomROMs da die jeweiligen Geräte zum Testen benötigt werden
	\end{itemize}
\end{frame}

\begin{frame}{Struktur der Software}
	\begin{itemize}[<+->]
		\item Android teilt sich in folgende Partitionen:
		\begin{itemize}[<+->]
			\item Bootloader 
			\note{ Lädt den Kernel und das schlußendliche System }
			\item Recovery
			\note{ Die Recovery ist ein "Rettungssystem", welches erweiterte Funktionen auf dem Gerät zur Verfügung stellen kann. (Backups, zusätzliche Pakete installieren, etc.) }
			\item System
			\note{ Das Betriebssystem an sich }
		\end{itemize}
		\item Dazu kommen kleinere Partitionen welche weitere Systemfunktionen garantieren.
	\end{itemize}
\end{frame}